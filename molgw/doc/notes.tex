\documentclass[aps,prb,reprint,showpacs]{revtex4-1}
\usepackage{graphicx}
\usepackage{latexsym}
\usepackage{amsmath}
\usepackage{amssymb}
\usepackage{amsfonts}
\usepackage{mathrsfs}
\usepackage{color}
\usepackage{verbatim}

%definitions
\def\br{\mathbf{r}}
\def\bR{\mathbf{R}}
\def\bq{\mathbf{q}}
\def\bk{\mathbf{k}}
\def\bG{\mathbf{G}}
\def\half{\frac{1}{2}}
\def\Rho{\mathrm{P}}
\def\h2o{\mathrm{H}_2\mathrm{O}}

\newcommand\refeq[1]{Eq.~(\ref{#1})}
\newcommand\refeqs[1]{Eqs.~(\ref{#1})}
\newcommand\wf[2]{\phi_{#1}^{#2}}


\begin{document}


\title{Implementation notes for \textsc{MOLGW}}
\author{Fabien Bruneval}
\affiliation{CEA, DEN, Service de Recherches de M\'etallurgie Physique, F-91191 Gif-sur-Yvette, France}

\maketitle






\section{Basic ingredients}

The wavefunctions are written as a linear combination of contracted Gaussian functions:
\begin{equation}
 \varphi_{i\sigma} (\br) = \sum_{\alpha} C_{\alpha i \sigma} \phi_\alpha(\br) .
\end{equation}

The Coulomb integrals are written as
\begin{equation}
 ( i j \sigma | k l \tau ) = \iint d\br d\br' \varphi_{i\sigma} (\br)  \varphi_{j\sigma} (\br) \frac{1}{|\br-\br'|}  \varphi_{k\tau}(\br') \varphi_{l\tau}(\br')
\end{equation}
The wavefunctions are real, so the conguate symbols are not required.



\section{BSE in transition space}

\begin{equation}
  [\chi_0^{-1}]_{ij \sigma}^{kl \tau}(\omega) =  
     \frac{ \omega - (\epsilon_{j\sigma} - \epsilon_{i\sigma} ) }
          {   f_{i\sigma}  - f_{j\sigma} } \delta_{ij} \delta_{kl} \delta_{\sigma\tau} 
\end{equation}

\begin{multline}
  [\chi^{-1}]_{ij \sigma}^{kl \tau}(\omega) =  
     \frac{ \omega - (\epsilon_{j\sigma} - \epsilon_{i\sigma} ) }
          {   f_{i\sigma}  - f_{j\sigma} } \delta_{ij} \delta_{kl} \delta_{\sigma\tau}  \\
    - ( ij \sigma | kl \tau )
    + \delta_{\sigma\tau} ( ik \sigma | jl \sigma) 
\end{multline}


\begin{multline}
( f_{i\sigma}  - f_{j\sigma} )  [\chi^{-1}]_{ij \sigma}^{kl \tau}(\omega) =
    [ \omega - ( \epsilon_{j\sigma} - \epsilon_{i\sigma} ) ] \delta_{ij} \delta_{kl} \delta_{\sigma\tau}  \\
    - ( f_{i\sigma}  - f_{j\sigma} ) [ ( ij \sigma | kl \tau ) + \delta_{\sigma\tau} ( ik \sigma | jl \sigma) ] 
\end{multline}


\begin{multline}
 [ H_\mathrm{BSE} ]_{ij \sigma}^{kl \tau} = 
      ( \epsilon_{j\sigma} - \epsilon_{i\sigma} ) \delta_{ij} \delta_{kl} \delta_{\sigma\tau}  \\
    + ( f_{i\sigma}  - f_{j\sigma} ) [ ( ij \sigma | kl \tau ) - \delta_{\sigma\tau} ( ik \sigma | jl \sigma) ] 
\end{multline}




\section{Symmetrization of the Bethe-Salpeter equation}


Let us write the TDHF equation.
The generalization to screened TDHF (=BSE) will come later.
\begin{subequations}
\begin{eqnarray}
 \label{eq:xy1}
 A X   & + B Y    &=    \Omega X \\
 \label{eq:xy2}
 -B X  & -A Y     &= \Omega Y
\end{eqnarray}
\end{subequations}
where
\begin{subequations}
\begin{eqnarray}
 A_{ij\sigma \, kl \tau} & = & \delta_{ik} \delta_{jl} \delta_{\sigma\tau} (\epsilon_{j\sigma} - \epsilon_{i\sigma}) \nonumber \\
                         &   & + ( i j \sigma | k l \tau ) - \delta_{\sigma\tau} ( i k | j l ) \\
 B_{ij\sigma \, kl \tau} & = & ( i j \sigma | k l \tau )  - \delta_{\sigma\tau} ( i k | j l ) 
\end{eqnarray}
\end{subequations}

By adding or subtracting Eqs.~(\ref{eq:xy1}),(\ref{eq:xy2}), one gets
\begin{subequations}
\begin{eqnarray}
 ( A + B ) ( X + Y ) & = &  \Omega ( X - Y ) \\
 ( A - B ) ( X - Y ) & = &  \Omega ( X + Y ) 
\end{eqnarray}
\end{subequations}
Combining the latter two equations yields an equation for $(X+Y)$
\begin{equation}
 ( A + B ) ( X+Y ) = \Omega^2 (A-B)^{-1} (X+Y) ,
\end{equation}
which can be further symmetrized by introducing
\begin{equation}
 Z = (A-B)^{-1/2} (X+Y) .
\end{equation}
Finally,
\begin{equation}
 (A-B)^{1/2}(A+B)(A-B)^{1/2} Z = \Omega^2 Z .
\end{equation}
The matrix we have to diagonalize is now symmetric and half smaller than the BSE matrix.

How to get $X$ and $Y$ from $Z$? 
\begin{subequations}
\begin{eqnarray}
  X + Y &=& (A-B)^{1/2} Z \\
  X - Y &=& \Omega (A-B)^{-1/2} Z .
\end{eqnarray}
\end{subequations}
Then adding or subtracting the latter two equations,
\begin{subequations}
\begin{eqnarray}
  X  &=&  \half \left[ (A-B)^{1/2} + \Omega (A-B)^{-1/2}  \right] Z  \\
  Y  &=&  \half \left[ (A-B)^{1/2} - \Omega (A-B)^{-1/2}  \right] Z .
\end{eqnarray}
\end{subequations}
























\end{document}








